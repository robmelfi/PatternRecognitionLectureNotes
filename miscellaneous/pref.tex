%%%%%%%%%%%%%%%%%%%%%% pref.tex %%%%%%%%%%%%%%%%%%%%%%%%%%%%%%%%%%%%%
%
% sample preface
%
% Use this file as a template for your own input.
%
%%%%%%%%%%%%%%%%%%%%%%%% Springer-Verlag %%%%%%%%%%%%%%%%%%%%%%%%%%

\preface

%% Please write your preface here
Questi appunti sono stati scritti durante il corso di Riconoscimento e Classificazione di Forme tenuto dal Professore Alfredo Petrosino nell’anno accademico 2008/2009. Non sono semplici appunti, ma un’accurata sbobinatura di registrazioni delle sue lezioni unite a parti tradotte e sintetizzate del libro “Duda Hart” \cite{duda}. 
A distanza di circa dieci anni rendo pubblico questo lavoro, non per meriti, ma per tramandare per iscritto le lezioni del professore ai futuri studenti, se esistono questi appunti è solo grazie al suo corso, ho avuto soltanto la pazienza di trascriverli e organizzarli per poter preparare al meglio l’esame finale. \\
Gli appunti sono open source (\url{https://github.com/robmelfi/PatternRecognitionLectureNotes}), cosi da correggere eventuali errori di battitura e migliorare parti non abbastanza curate da parte mia. 
Il professore ha insegnato che la perfezione non esiste, ma un lavoro può essere sempre migliorato per tendere alla perfezione, questi appunti devono essere un punto di partenza per essere ampliati lasciando inalterate e mettendo in evidenza le parole del professore. 



%% Please "sign" your preface

\begin{flushright}\noindent
\hfill {\it Roberto Melfi}\\
\end{flushright}
\vspace{0.5cm}

\noindent Il ruolo di un professore è quello di essere guida e di insegnare la vita, oltre che la disciplina.
Gli studenti dell’Università Parthenope che hanno conosciuto il professore Alfredo Petrosino possono dire di aver avuto un’insegnante che ha ricoperto a pieno questo ruolo.
Grazie a lui abbiamo creato un gruppo di studio tra le scrivanie del CVPRLab che andava oltre i banchi universitari.
Durante tutto il percorso accademico abbiamo trascorso dei momenti intensi che ci han forgiato rendendoci le persone migliori che siamo oggi. Non gli saremo mai abbastanza riconoscenti per questo!
La sua prematura scomparsa è stata un fulmine a ciel sereno che ha riunito sotto lo stesso tetto, tutte le persone che lo conoscevano ed avevano lavorato con lui. Anche in questo momento è stato speciale dato ci siamo riuniti dopo anni in cui ognuno di noi ha intrapreso la sua strada.
Grazie di tutto prof!\\

\begin{flushright}\noindent
\hfill {\it Luca Russo}\\
\end{flushright}
\vspace{0.5cm}
“ti aspetterò comunque fuori alla porta 429 ... sarai semplicemente in ritardo ...”, il saluto del suo caro allievo \emph{Enrico Balestrieri}.

\vspace{0.5cm}
Napoli, Gennaio 2020





