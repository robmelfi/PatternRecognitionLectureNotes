%%%%%%%%%%%%%%%%%%%%%%%% referenc.tex %%%%%%%%%%%%%%%%%%%%%%%%%%%%%%
% sample references
% "computer science"
%
% Use this file as a template for your own input.
%
%%%%%%%%%%%%%%%%%%%%%%%% Springer-Verlag %%%%%%%%%%%%%%%%%%%%%%%%%%

%
% BibTeX users please use
% \bibliographystyle{}
% \bibliography{}
%
% Non-BibTeX users please use
\begin{thebibliography}{99.}
%
% and use \bibitem to create references.
%
% Use the following syntax and markup for your references
%
% Monographs

\bibitem{contribution} Audiolezioni Corso Riconoscimento e Classificazione di Forme A.A, 2008/2009, Alfredo Petrosino

\bibitem{book}  Richard O. Duda, Peter E. Hart, David G. Stork, Pattern Classification

\bibitem{book}  Calcolo delle probabilità - Seymour LIPSCHUTZ (collana schaum) - ETAS Libri


%\bibitem{monograph} Kajan E (2002)
%Information technology encyclopedia and acronyms. Springer, Berlin
%Heidelberg New York
%
%% Contributed Works
%\bibitem{contribution} Broy M (2002) Software engineering -- From
%auxiliary to key technologies. In: Broy M, Denert E (eds)
%Software Pioneers. Springer, Berlin Heidelberg New York
%
%% Journal
%\bibitem{journal} Che M, Grellmann W, Seidler S (1997)
%Appl Polym Sci 64:1079--1090
%
%% Theses
%\bibitem{thesis} Ross DW (1977) Lysosomes and storage diseases. MA
%Thesis, Columbia University, New York

\end{thebibliography}
